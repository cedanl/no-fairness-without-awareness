% Options for packages loaded elsewhere
% Options for packages loaded elsewhere
\PassOptionsToPackage{unicode}{hyperref}
\PassOptionsToPackage{hyphens}{url}
\PassOptionsToPackage{dvipsnames,svgnames,x11names}{xcolor}
%
\documentclass[
  11pt,
  a4paper,
]{article}
\usepackage{xcolor}
\usepackage[margin=1in]{geometry}
\usepackage{amsmath,amssymb}
\setcounter{secnumdepth}{-\maxdimen} % remove section numbering
\usepackage{iftex}
\ifPDFTeX
  \usepackage[T1]{fontenc}
  \usepackage[utf8]{inputenc}
  \usepackage{textcomp} % provide euro and other symbols
\else % if luatex or xetex
  \usepackage{unicode-math} % this also loads fontspec
  \defaultfontfeatures{Scale=MatchLowercase}
  \defaultfontfeatures[\rmfamily]{Ligatures=TeX,Scale=1}
\fi
\usepackage{lmodern}
\ifPDFTeX\else
  % xetex/luatex font selection
\fi
% Use upquote if available, for straight quotes in verbatim environments
\IfFileExists{upquote.sty}{\usepackage{upquote}}{}
\IfFileExists{microtype.sty}{% use microtype if available
  \usepackage[]{microtype}
  \UseMicrotypeSet[protrusion]{basicmath} % disable protrusion for tt fonts
}{}
\makeatletter
\@ifundefined{KOMAClassName}{% if non-KOMA class
  \IfFileExists{parskip.sty}{%
    \usepackage{parskip}
  }{% else
    \setlength{\parindent}{0pt}
    \setlength{\parskip}{6pt plus 2pt minus 1pt}}
}{% if KOMA class
  \KOMAoptions{parskip=half}}
\makeatother
% Make \paragraph and \subparagraph free-standing
\makeatletter
\ifx\paragraph\undefined\else
  \let\oldparagraph\paragraph
  \renewcommand{\paragraph}{
    \@ifstar
      \xxxParagraphStar
      \xxxParagraphNoStar
  }
  \newcommand{\xxxParagraphStar}[1]{\oldparagraph*{#1}\mbox{}}
  \newcommand{\xxxParagraphNoStar}[1]{\oldparagraph{#1}\mbox{}}
\fi
\ifx\subparagraph\undefined\else
  \let\oldsubparagraph\subparagraph
  \renewcommand{\subparagraph}{
    \@ifstar
      \xxxSubParagraphStar
      \xxxSubParagraphNoStar
  }
  \newcommand{\xxxSubParagraphStar}[1]{\oldsubparagraph*{#1}\mbox{}}
  \newcommand{\xxxSubParagraphNoStar}[1]{\oldsubparagraph{#1}\mbox{}}
\fi
\makeatother


\usepackage{longtable,booktabs,array}
\usepackage{calc} % for calculating minipage widths
% Correct order of tables after \paragraph or \subparagraph
\usepackage{etoolbox}
\makeatletter
\patchcmd\longtable{\par}{\if@noskipsec\mbox{}\fi\par}{}{}
\makeatother
% Allow footnotes in longtable head/foot
\IfFileExists{footnotehyper.sty}{\usepackage{footnotehyper}}{\usepackage{footnote}}
\makesavenoteenv{longtable}
\usepackage{graphicx}
\makeatletter
\newsavebox\pandoc@box
\newcommand*\pandocbounded[1]{% scales image to fit in text height/width
  \sbox\pandoc@box{#1}%
  \Gscale@div\@tempa{\textheight}{\dimexpr\ht\pandoc@box+\dp\pandoc@box\relax}%
  \Gscale@div\@tempb{\linewidth}{\wd\pandoc@box}%
  \ifdim\@tempb\p@<\@tempa\p@\let\@tempa\@tempb\fi% select the smaller of both
  \ifdim\@tempa\p@<\p@\scalebox{\@tempa}{\usebox\pandoc@box}%
  \else\usebox{\pandoc@box}%
  \fi%
}
% Set default figure placement to htbp
\def\fps@figure{htbp}
\makeatother





\setlength{\emergencystretch}{3em} % prevent overfull lines

\providecommand{\tightlist}{%
  \setlength{\itemsep}{0pt}\setlength{\parskip}{0pt}}



 


\makeatletter
\@ifpackageloaded{caption}{}{\usepackage{caption}}
\AtBeginDocument{%
\ifdefined\contentsname
  \renewcommand*\contentsname{Table of contents}
\else
  \newcommand\contentsname{Table of contents}
\fi
\ifdefined\listfigurename
  \renewcommand*\listfigurename{List of Figures}
\else
  \newcommand\listfigurename{List of Figures}
\fi
\ifdefined\listtablename
  \renewcommand*\listtablename{List of Tables}
\else
  \newcommand\listtablename{List of Tables}
\fi
\ifdefined\figurename
  \renewcommand*\figurename{Figure}
\else
  \newcommand\figurename{Figure}
\fi
\ifdefined\tablename
  \renewcommand*\tablename{Table}
\else
  \newcommand\tablename{Table}
\fi
}
\@ifpackageloaded{float}{}{\usepackage{float}}
\floatstyle{ruled}
\@ifundefined{c@chapter}{\newfloat{codelisting}{h}{lop}}{\newfloat{codelisting}{h}{lop}[chapter]}
\floatname{codelisting}{Listing}
\newcommand*\listoflistings{\listof{codelisting}{List of Listings}}
\makeatother
\makeatletter
\makeatother
\makeatletter
\@ifpackageloaded{caption}{}{\usepackage{caption}}
\@ifpackageloaded{subcaption}{}{\usepackage{subcaption}}
\makeatother
\usepackage{bookmark}
\IfFileExists{xurl.sty}{\usepackage{xurl}}{} % add URL line breaks if available
\urlstyle{same}
\hypersetup{
  pdftitle={De uitkomsten van de kansengelijkheidanalysis},
  colorlinks=true,
  linkcolor={blue},
  filecolor={Maroon},
  citecolor={Blue},
  urlcolor={Blue},
  pdfcreator={LaTeX via pandoc}}


\title{De uitkomsten van de kansengelijkheidanalysis}
\usepackage{etoolbox}
\makeatletter
\providecommand{\subtitle}[1]{% add subtitle to \maketitle
  \apptocmd{\@title}{\par {\large #1 \par}}{}{}
}
\makeatother
\subtitle{B International Business Administration VT}
\author{}
\date{}
\begin{document}
\maketitle


\subsection{Onderzoek naar
kansengelijkheid}\label{onderzoek-naar-kansengelijkheid}

Het lectoraat Learning Technology \& Analytics (LTA) van De Haagse
Hogeschool heeft tot doel kansengelijkheid voor studenten te verhogen
met behulp van learning analytics en inzet van learning technology.

Het lectoraat heeft een onderzoeksmethode ontwikkeld om te kunnen
analyseren of er sprake is van bias in studiedata in relatie tot het
succes van studenten, wat een indicatie kan zijn van een gebrek aan
kansengelijkheid. Deze methode gebruikt prognosemodellen op basis van
machine learning. Een prognosemodel is dus niet een doel op zich, maar
het instrument voor een analyse van kansengelijkheid, ook wel een
\emph{fairness} analyse genoemd.

Over deze methode heeft de lector, Dr.~Theo Bakker, zijn intreerede
uitgesproken op 21 november 2024, getiteld:
`\href{https://zenodo.org/records/14204674}{No Fairness without
Awareness. Toegepast onderzoek naar kansengelijkheid in het hoger
onderwijs. Intreerede lectoraat Learning Technology \& Analytics.}'
{[}@Bakker.2024-intreerede{]}. Zie voor een verdere toelichting op het
gehele onderzoeksprogramma:
`\href{https://www.dehaagsehogeschool.nl/onderzoek/kenniscentra/no-fairness-without-awareness}{No
Fairness without Awareness}'.

\subsection{Conclusies}\label{conclusies}

\begin{enumerate}
\def\labelenumi{\arabic{enumi}.}
\tightlist
\item
  \textbf{geslacht}: Er is een negatieve bias voor vrouwen (V: N = 408,
  31,7\%).
\item
  \textbf{vooropleiding}: Er is een negatieve bias voor studenten die
  instromen met een diploma in het hoger onderwijs (HO: N = 396,
  30,7\%). Er is een positieve bias voor studenten met een buitenlands
  diploma (BD: N = 92, 7,1\%).
\item
  \textbf{aansluiting}: Er is een negatieve bias voor interne switchers
  (Switch intern: N = 91, 7,1\%), studenten die twee of meer studies
  volgen (2e Studie: N = 47, 3,6\%). Er is een positieve bias voor
  studenten met een overige aansluiting (Overig: N = 176, 13,7\%).
\end{enumerate}

\newpage

\begin{center}
\pandocbounded{\includegraphics[keepaspectratio]{../output/result_table.png}}
\end{center}

\textbf{Toelichting:}

\begin{itemize}
\item
  \providecolor{sqfillA84268}{HTML}{A84268}
  \providecolor{sqborderA9A9A9}{HTML}{A9A9A9}

  \begingroup\setlength{\fboxsep}{1pt}\fcolorbox{sqborderA9A9A9}{sqfillA84268}{\rule{0pt}{12.0pt}\rule{12.0pt}{0pt}}\endgroup  Negatieve
  bias.
\item
  \providecolor{sqfill9DBF9E}{HTML}{9DBF9E}
  \providecolor{sqborderA9A9A9}{HTML}{A9A9A9}

  \begingroup\setlength{\fboxsep}{1pt}\fcolorbox{sqborderA9A9A9}{sqfill9DBF9E}{\rule{0pt}{12.0pt}\rule{12.0pt}{0pt}}\endgroup  Positieve
  bias.
\item
  \providecolor{sqfillFCB97D}{HTML}{FCB97D}
  \providecolor{sqborderA9A9A9}{HTML}{A9A9A9}

  \begingroup\setlength{\fboxsep}{1pt}\fcolorbox{sqborderA9A9A9}{sqfillFCB97D}{\rule{0pt}{12.0pt}\rule{12.0pt}{0pt}}\endgroup  Bias,
  maar de aantallen studenten zijn te laag om conclusies over een
  negatieve of positieve bias aan te verbinden.
\item
  \providecolor{sqfillE5E5E5}{HTML}{E5E5E5}
  \providecolor{sqborderA9A9A9}{HTML}{A9A9A9}

  \begingroup\setlength{\fboxsep}{1pt}\fcolorbox{sqborderA9A9A9}{sqfillE5E5E5}{\rule{0pt}{12.0pt}\rule{12.0pt}{0pt}}\endgroup  De
  bevoorrechte groep. Hiervan dient een eventuele bias nader bepaald te
  worden (NTB = Nader te bepalen). Dit is het geval als alle overige
  groepen binnen een variabelen een bias hebben.
\item
  We hanteren een minimum van 15 studenten per categorie binnen een
  variabele om een oordeel te geven.
\end{itemize}




\end{document}
